%! Author = Ian
%! Date = 3/17/2024

% Preamble
\documentclass{article}

% Packages
\usepackage{amsmath}
\usepackage{fancyhdr}
\usepackage[top=2cm, left=2cm, right=2cm, bottom=2cm]{geometry}
\usepackage{multicol}
\usepackage{amsfonts}
\usepackage{amssymb}
\setlength{\columnsep}{0.5cm}

\author{Ian Chen}
\date{\today}

% Header
\fancyhf{}
\fancyhead[L]{Ian Chen}
\fancyhead[C]{CS331 Exam 2 Notes}
\fancyhead[R]{\today}
\pagestyle{fancy}

% Document
\begin{document}
    \begin{multicols*}{2}
        \subsection*{Proof of Correctness}
        Total Correctness: Termination and Partial Correctness\\
        Partial Correctness: Loop invariants and induction\\
        Loop Invariant: A property that holds before and after each iteration of a loop\\
        Initialization: The loop invariant holds before the first iteration\\
        Maintenance: If the loop invariant holds before an iteration, it holds after the iteration\\
        Termination: When loop terminates, invariant gives useful property to show the algorithm is
        correct\\
        Iterative: Usually loop invariants\\
        Recursive: Usually induction\\
        D\&C: Show recurrence is optimal inductively by showing sub-problems generate optimal
        solutions\\
        DP: Show recurrence is optimal by description of optimal substructure then show algorithm
        implements the recurrence
        \subsection*{Complexity}
        $a^{\log_b x} = x^{\log_b a}$\\
        $\log_b x = \frac{\log_c x}{\log_c b}$\\
        $\log_b M \cdot N = \log_b M + \log_b N$\\
        $\log_b \frac{M}{N} = \log_b M - \log_b N$\\
        $\log_b M^k = k\log_b M$\\
        Big Oh: $f(n)$ is $O(g(n))$ if $f(n) \leq cg(n)$ for $n \geq n_0 : c, n_0 > 0$\\
        Big Omega: $f(n)$ is $\Omega(g(n))$ if $f(n) \geq cg(n)$ for $n \geq n_0 : c, n_0 > 0$\\
        Big Theta: $f(n)$ is $\Theta(g(n))$ if $f(n)$ is $O(g(n))$ and $f(n)$ is $\Omega(g(n))$\\
        Little Oh: Strict Big Oh\\
        Little Omega: Strict Big Omega\\
        $\lim_{n\to\infty}\frac{f(n)}{g(n)}$:\\
        0 if $f(n)$ is $o(g(n))$, $\infty$ if $f(n)$ is $\omega(g(n))$\\
        $< \infty$ if $f(n)$ is $O(g(n))$, $> 0$ if $f(n)$ is $\Omega(g(n))$\\
        $0 < \infty$ if $f(n)$ is $\Theta(g(n))$\\
        Growth Rates: $1 < \log(n) < \sqrt{n} < n < n\log(n) < n^2 < n^c < 2^n < c^n < n! < n^n$\\
        Harmonic: $\sum_{k=1}^{n} \frac{1}{k} \sim \ln n$\\
        Triangular: $\sum_{k=1}^{n} k = \frac{n(n + 1)}{2} \sim \frac{n^2}{2}$\\
        Squares: $\sum_{k=1}^{n} k^2 \sim \frac{n^3}{3}$\\
        Geometric: $\sum_{k=0}^{n} ar^k = \frac{a(r^{n+1} - 1)}{r - 1}$\\
        Stirling's Approximation: $\log_2(n!) \sim n\log_2 n$\\
        Master's Theorem: $T(n) = aT(\frac{n}{b}) + f(n)$, $\epsilon > 0$, $a,b$ constant, $f(n
        ) \geq 0$\\
        $f(n) = O(n^{\log_b (a) - \epsilon}) \to T(n) = \Theta(n^{\log_b a})$\\
        $f(n) = \Theta(n^{\log_b a}\log^k n) \wedge k \geq 0 \to T(n) =
        \Theta(n^{\log_b a}\log^{k+1}n)$\\
        $f(n) = \Omega(n^{log_b (a) + \epsilon}) \wedge af(\frac{n}{b}) \leq cf(n)$ for some $c
        < 1 \to T(n) = \Theta(f(n))$
        \subsection*{Divide and Conquer}
        Divide: Break problem into smaller independent sub-problems\\
        Conquer/Combine: Solve sub-problems recursively and combine
        \subsubsection*{Recurrence Relation}
        A function defined in terms of itself(recursively)\\
        Analysis of D\&C generally involves a recurrence relation\\
        1,2,3 Method: $T(n) = \sum^{levels}$time per level or
        $\sum^{levels-1}$time per level + (base case value $\cdot$ \# of leaves)
        \subsubsection*{Merge Sort}
        $T(n) = 2T(\frac{n}{2}) + n-1$, $T(1) = 0$\\
        \begin{tabular}{c|c|c}
            Level  & Problem Size        & Total Time               \\
            \hline
            0      & $n$                 & $n$                      \\
            1      & $\frac{n}{2}$       & $2\frac{n}{2} = n$       \\
            2      & $\frac{n}{4}$       & $4\frac{n}{4} = n$       \\
            \vdots & \vdots              & \vdots                   \\
            $k$    & $\frac{n}{2^k} = 1$ & $2^{k}\frac{n}{2^k} = n$ \\
        \end{tabular}\\
        $\to (\sum\limits_{i=0}^{k-1} n) + 0 \cdot 2^{log_2 n} \to \sum\limits_{i=0}^{log_2 n -
        1} n \to n\log_2 n$\\
        or $\to \sum\limits_{i=0}^{k} n \to \sum\limits_{i=0}^{log_2 n} n \to n\log_2 n$
        \subsubsection*{Closest Pair}
        D\&C by splitting plane in half by median then checking in rectangles\\
        $O(n\log n)$
        \subsection*{Dynamic Programming}
        Overlapping and dependant sub-problems
        \subsubsection*{Weighted Interval Scheduling}
        j- intervals sorted by latest finishing time\\
        p(j)- interval that immediately precedes j without overlap\\
        OPT(j) = max(OPT(j - 1), $v_j$ + OPT(p(j)))\\
        Can use recursive, top-down approach(memoization) or iterative, bottom-up(tabulation)\\
        n = number of intervals\\
        $O(n\log n)$ to sort, $O(n)$ for algorithm
        \subsubsection*{Subset Sum}
        i- index of item, W- max weight\\
        OPT(i, W) = OPT(i - 1, W) if $W < w_i$\\
        else max($w_i$ + OPT(i - 1, W - $w_i$), OPT(i - 1, W))\\
        n = number of items, W = sum\\
        $O(n \cdot W)$, pseudo-polynomial
        \subsubsection*{0/1 Knapsack}
        Maximize value instead of weight\\
        i- index of item, v- value, W- max weight\\
        OPT(i, W) = OPT(i - 1, W) if $W < w_i$\\
        else max($v_i$ + OPT(i - 1, W - $w_i$), OPT(i - 1, W))
        \subsubsection*{Sequence Alignment}
        $\delta$- gap cost, $\alpha$- alignment cost\\
        OPT(i, j) = min($\alpha_{x_iy_j}$ + OPT(i - 1, j - 1),
        $\delta$ + OPT(i - 1, j), $\delta$ + OPT(i, j - 1))\\
        m, n = length of strings\\
        $O(mn)$
        \subsubsection*{Bellman-Ford}
        i- number of usable edges, v- start node, w- intermediate node\\
        OPT(i, v) = min(OPT(i - 1, v), min$_{w\in V}$(OPT(i - 1, w) + $c_{vw}$))\\
        m = number of edges, n = number of nodes\\
        $O(mn)$\\
        OPT(n, v) $\neq$ OPT(n - 1, v) $\to$ path has negative cycle
        \subsection*{Network Flow}
        Source(s): Only outgoing edges\\
        Sink(t): Only incoming edges\\
        Capacity($c_e$): $\in \mathbb{N}$\\
        $f(e)$: Flow through edge e, $\geq$0\\
        $v(f)$: Value of flow f, $\sum_{e\ out\ s} f(e)$
        \subsubsection*{Ford-Fulkerson}
        Implementation of max-flow problem\\
        (1) Find simple $s - t$ path and set flow to bottleneck\\
        (2) Build residual graph $G_f$ with backward edge if $f(e) > 0$ and forward edge if
        $f(e) < c_e$\\
        (3) Repeat till no simple $s - t$ path in $G_f$:\\
        Call bottleneck value $b$\\
        Augment G by incrementing forward edges by $f(e) += b$ and backward edges by $f(e) -= b$\\
        Update $G_f$\\
        (4) No $s - t$ path in $G_f$ after algorithm terminates\\
        m = number of edges, C = max possible flow\\
        $O(mC)$\\
        \subsubsection*{Max Flow/Min Cut}
        Cut: Partition of nodes\\
        A = subset with s, B = subset with t\\
        Cut Capacity: $c(A, B) = \sum_{e\ out\ A} c_e$\\
        $v(f) = f^{out}(A) - f^{in}(A) = f^{in}(B) - f^{out}(B)$\\
        Min cut = Max flow
    \end{multicols*}
\end{document}