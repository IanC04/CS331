\documentclass[12pt]{article}
\usepackage{amsmath,amsfonts,amsthm,amssymb}
\usepackage{setspace}
\usepackage{fancyhdr}
\usepackage{lastpage}
\usepackage{extramarks}
\usepackage[ruled,vlined]{algorithm2e}
\usepackage{chngpage}
\usepackage{soul,color, xcolor}
\usepackage{graphicx,float,wrapfig}
\usepackage{ listings}
\usepackage{enumitem}
\usepackage{tikz}
\newcommand{\Class}{ \normalsize CS 331: Algorithms and Complexity (Spring 2024)\\
\small    Unique Number: 50930, 50935 50940, 50945
}


\newenvironment{solution}{
    \color{blue} \textbf{Solution:}
    }
    %\newcommand{\ClassInstructor}{Fares}

        \def\changemargin#1#2{\list{}{\rightmargin#2\leftmargin#1}\item[]}
\let\endchangemargin=\endlist
% Homework Specific Information. Change it to your own
\newcommand{\Title}{Assignment 3 - Solution}
\newcommand{\DueDate}{Thursday, 8 Febrauary, by 11.59pm}
\newcommand{\StudentName}{}
\newcommand{\StudentClass}{}
\newcommand{\StudentNumber}{}

% In case you need to adjust margins:
\topmargin=-0.45in      %
\evensidemargin=0in     %
\oddsidemargin=0in      %
\textwidth=6.5in        %
\textheight=9.0in       %
\headsep=0.25in         %

% Setup the header and footer
\pagestyle{fancy}                                                       %
\lhead{\StudentName}                                                 %
\chead{\Title}  %
\rhead{\firstxmark}                                                     %
\lfoot{\lastxmark}                                                      %
\cfoot{}                                                                %
\rfoot{Page\ \thepage\ of\ \protect\pageref{LastPage}}                          %
\renewcommand\headrulewidth{0.4pt}                                      %
\renewcommand\footrulewidth{0.4pt}                                      %

%%%%%%%%%%%%%%%%%%%%%%%%%%%%%%%%%%%%%%%%%%%%%%%%%%%%%%%%%%%%%
% Some tools
\newcommand{\enterProblemHeader}[1]{\nobreak\extramarks{#1}{#1 continued on next page\ldots}\nobreak%
\nobreak\extramarks{#1 (continued)}{#1 continued on next page\ldots}\nobreak}%
\newcommand{\exitProblemHeader}[1]{\nobreak\extramarks{#1 (continued)}{#1 continued on next page
\ldots}\nobreak%
\nobreak\extramarks{#1}{}\nobreak}%

\newcommand{\homeworkProblemName}{}%
\newcounter{homeworkProblemCounter}%
\newenvironment{homeworkProblem}[1][Problem \arabic{homeworkProblemCounter}]%
{\stepcounter{homeworkProblemCounter}%
\renewcommand{\homeworkProblemName}{#1}%
\section*{\homeworkProblemName}%
\enterProblemHeader{\homeworkProblemName}}%
{\exitProblemHeader{\homeworkProblemName}}%

\newcommand{\homeworkSectionName}{}%
\newlength{\homeworkSectionLabelLength}{}%
\newenvironment{homeworkSection}[1]%
{% We put this space here to make sure we're not connected to the above.

    \renewcommand{\homeworkSectionName}{#1}%
    \settowidth{\homeworkSectionLabelLength}{\homeworkSectionName}%
    \addtolength{\homeworkSectionLabelLength}{0.25in}%
    \changetext{}{-\homeworkSectionLabelLength}{}{}{}%
    \subsection*{\homeworkSectionName}%
    \enterProblemHeader{\homeworkProblemName\ [\homeworkSectionName]}}%
    {\enterProblemHeader{\homeworkProblemName}%

    % We put the blank space above in order to make sure this margin
    % change doesn't happen too soon.
    \changetext{}{+\homeworkSectionLabelLength}{}{}{}}%

\newcommand{\Answer}{\ \\\textbf{Answer:} }
\newcommand{\Acknowledgement}[1]{\ \\{\bf Acknowledgement:} #1}

%%%%%%%%%%%%%%%%%%%%%%%%%%%%%%%%%%%%%%%%%%%%%%%%%%%%%%%%%%%%%


%%%%%%%%%%%%%%%%%%%%%%%%%%%%%%%%%%%%%%%%%%%%%%%%%%%%%%%%%%%%%
% Make title
\newcommand{\maxpt}[1]{\ifthenelse{\equal{#1}{1}}{\textbf{(#1 pt)}}{\textbf{(#1 pts)}}}
\def \solcolor{blue}
\title{\textmd{\bf \Class\\ \Title}\\\vspace{0.1in}\small{Due\ on\ \DueDate}}
\date{}
\author{\textbf{\StudentName}\ \ \StudentClass\ \ \StudentNumber}
%%%%%%%%%%%%%%%%%%%%%%%%%%%%%%%%%%%%%%%%%%%%%%%%%%%%%%%%%%%%%

\begin{document}
    \maketitle \thispagestyle{empty}

    %        \begin{figure}[h]
    %            \centering
    %            \includegraphics[width=0.75\textwidth]{../HonorCode}
    %        \end{figure}


    %%%%%%%%%%%%%%%%%%%%%%%%%%%%%%%%%%%%%%%%%%%%%%%%%%%%%%%%%%%%%
    % Begin edit from here


    \def\changemargin#1#2{\list{}{\rightmargin#2\leftmargin#1}\item[]}
    \let\endchangemargin=\endlist

    \begin{homeworkProblem}[Problem 1: Short Answer Section]
        \maxpt{10} True or false. If true, briefly justify, otherwise, provide a counter example
        . When justifying, restrict answers to no more than a few sentences.
        \begin{enumerate}
            \item \maxpt{1}
            True, a greedy option picks the best option at each step, without considering the
            future.
            \item \maxpt{2}
            True, you argue that the differences between the two algorithms are irrelevant to the
            quality of the solution.
            \item \maxpt{2}
            True, since the shortest path is the path with the fewest edges.
            \item \maxpt{2}
            False, Counterexample:
            \begin{verbatim}
(1, 2), (2, 3), (1, 3)
Pick (1, 3) since it has the largest time interval
Since it only conflicts with (1, 2) and (2, 3), we can pick it
However, the optimal solution is to pick (1, 2) and (2, 3)
            \end{verbatim}
            \item \maxpt{3}
            No, the shortest path is not necessarily the path with the fewest edges, Counterexample:
            \begin{verbatim}
(a, b) = 1; (b, c) = 1; (c, d) = 1; (a, d) = 4
Shortest path: a -> b -> c -> d with weight 3
Increment all edges by 1, we get: a -> b -> c -> d with weight 6
However, the shortest path is a -> d with weight 5
            \end{verbatim}
        \end{enumerate}
    \end{homeworkProblem}

    \begin{homeworkProblem}
        \textbf{(10 points)}
        I will denote tasks as (p(i), t(i)) where p(i) is the value of the task and t(i) is the
        duration of the task.
        \begin{enumerate}
            \item (Smallest duration first) Pick task $i$ that has the minimum duration $t(i)$, or
            \begin{proof}
                This is not optimal.\newline
                Counterexample:
                \begin{verbatim}
(1, 1), (10, 2)
Pick (1, 1) first, then (10, 2)
This yields (1 * 1) + (10 * 3) = 31
However, the optimal solution is to pick (10, 2) first, then (1, 1)
This yields (10 * 2) + (1 * 3) = 23
                \end{verbatim}
            \end{proof}
            \item (Most valuable first) Pick task $i$ that has maximum $p(i)$, or
            \begin{proof}
                This is not optimal.\newline
                Counterexample:
                \begin{verbatim}
(1, 1), (2, 3)
Pick (2, 3) first, then (1, 1)
This yields (2 * 3) + (1 * 4) = 10
However, the optimal solution is to pick (1, 1) first, then (2, 3)
This yields (1 * 1) + (2 * 4) = 9
                \end{verbatim}
            \end{proof}
            \item (Maximum time-scaled value first) Pick task $i$ that has maximum $p(i)/t(i)$.
            \begin{proof}
                This is optimal.
                % TODO Answer
            \end{proof}
        \end{enumerate}
    \end{homeworkProblem}

    \begin{homeworkProblem}
        \textbf{(10 points)}
        Yes
        % TODO Answer
    \end{homeworkProblem}

\end{document}
%%% Local Variables:
%%% mode: latex
%%% TeX-master: t
%%% End:
