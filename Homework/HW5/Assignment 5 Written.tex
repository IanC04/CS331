\documentclass[12pt]{article}
\usepackage{amsmath,amsfonts,amsthm,amssymb}
\usepackage{setspace}
\usepackage{fancyhdr}
\usepackage{lastpage}
\usepackage{extramarks}
\usepackage[ruled,vlined]{algorithm2e}
\usepackage{chngpage}
\usepackage{soul,color}
\usepackage{graphicx,float,wrapfig}
\usepackage{listings}
\lstset{
    basicstyle=\ttfamily,
    mathescape
}
\usepackage{hyperref}

\RequirePackage{amssymb, amsfonts, amsmath, latexsym, verbatim, xspace, setspace}
\RequirePackage{tikz, pgflibraryplotmarks}
%\usepackage{algorithm}
\usepackage{subcaption}
\usepackage{algorithmicx}
\usepackage[noend]{algpseudocode}
% honor
\usepackage{booktabs}
\usepackage{array}% http://ctan.org/pkg/array
\newcolumntype{M}{>{\centering\arraybackslash}m{\dimexpr.25\linewidth-2\tabcolsep}}
%

\usepackage{enumitem}
\usepackage[parfill]{parskip} % put space between paragraphs
\setlist{listparindent=\parindent} % make paragraphs in lists normal
\usepackage{xcolor}

\newcommand{\Class}{ \normalsize CS 331: Algorithms and Complexity (Spring 2024)\\
\small    Unique Number: 50930, 50935 50940, 50945
}


\def\changemargin#1#2{\list{}{\rightmargin#2\leftmargin#1}\item[]}
\let\endchangemargin=\endlist

%\newcommand{\ClassInstructor}{Fares}
%\newcommand{\ClassInstructor}{Fares}

\def\changemargin#1#2{\list{}{\rightmargin#2\leftmargin#1}\item[]}
\let\endchangemargin=\endlist


%\newcommand{\ClassInstructor}{Fares}
% Homework Specific Information. Change it to your own
\newcommand{\Title}{Assignment 5}
\newcommand{\DueDate}{Tuesday, 5 March, by 11.59pm}

\newcommand{\StudentName}{}
\newcommand{\StudentClass}{}
\newcommand{\StudentNumber}{}

% In case you need to adjust margins:
\topmargin=-0.45in      %
\evensidemargin=0in     %
\oddsidemargin=0in      %
\textwidth=6.5in        %
\textheight=9.0in       %
\headsep=0.25in         %

% Setup the header and footer
\pagestyle{fancy}                                                       %
\lhead{\StudentName}                                                 %
\chead{\Title}  %
\rhead{\firstxmark}                                                     %
\lfoot{\lastxmark}                                                      %
\cfoot{}                                                                %
\rfoot{Page\ \thepage\ of\ \protect\pageref{LastPage}}                          %
\renewcommand\headrulewidth{0.4pt}                                      %
\renewcommand\footrulewidth{0.4pt}                                      %

%%%%%%%%%%%%%%%%%%%%%%%%%%%%%%%%%%%%%%%%%%%%%%%%%%%%%%%%%%%%%
% Some tools
\newcommand{\enterProblemHeader}[1]{\nobreak\extramarks{#1}{#1 continued on next page\ldots}\nobreak%
\nobreak\extramarks{#1 (continued)}{#1 continued on next page\ldots}\nobreak}%
\newcommand{\exitProblemHeader}[1]{\nobreak\extramarks{#1 (continued)}{#1 continued on next page
\ldots}\nobreak%
\nobreak\extramarks{#1}{}\nobreak}%

%\definecolor{problemStatementColor}{gray}{0.4}
%\definecolor{problemSolutionColor}{gray}{0}
\newcommand{\homeworkProblemName}{}%
\newcounter{homeworkProblemCounter}%
\newenvironment{homeworkProblem}[1][Problem \arabic{homeworkProblemCounter}]%
{\stepcounter{homeworkProblemCounter}%
\renewcommand{\homeworkProblemName}{#1}%
\section*{\homeworkProblemName}%
\enterProblemHeader{\homeworkProblemName}
%\color{problemStatementColor}
}%
{\exitProblemHeader{\homeworkProblemName}}%

\newcommand{\homeworkSectionName}{}%
\newlength{\homeworkSectionLabelLength}{}%
\newenvironment{homeworkSection}[1]%
{% We put this space here to make sure we're not connected to the above.

    \renewcommand{\homeworkSectionName}{#1}%
    \settowidth{\homeworkSectionLabelLength}{\homeworkSectionName}%
    \addtolength{\homeworkSectionLabelLength}{0.25in}%
    \changetext{}{-\homeworkSectionLabelLength}{}{}{}%
    \subsection*{\homeworkSectionName}%
    \enterProblemHeader{\homeworkProblemName\ [\homeworkSectionName]}}%
    {\enterProblemHeader{\homeworkProblemName}%

    % We put the blank space above in order to make sure this margin
    % change doesn't happen too soon.
    \changetext{}{+\homeworkSectionLabelLength}{}{}{}}%

\newcommand{\partialSolution}{\textbf{Partial solution:} }
\newcommand{\solution}{\textbf{Solution:} }
\newcommand{\Acknowledgement}[1]{\ \\{\bf Acknowledgement:} #1}

\newcommand{\maxpt}[1]{\ifthenelse{\equal{#1}{1}}{\textbf{(#1 pt)}}{\textbf{(#1 pts)}}}
%%%%%%%%%%%%%%%%%%%%%%%%%%%%%%%%%%%%%%%%%%%%%%%%%%%%%%%%%%%%%


\DeclareMathOperator{\opt}{OPT}
\setlist[enumerate,1]{label=\bfseries(\alph*)}
%%%%%%%%%%%%%%%%%%%%%%%%%%%%%%%%%%%%%%%%%%%%%%%%%%%%%%%%%%%%%
% Make title
\title{\textmd{\bf \Class\\ \Title}\\\vspace{0.1in}\small{Due\ on\ \DueDate}}
\date{}
\author{\textbf{\StudentName}\ \ \StudentClass\ \ \StudentNumber}
%%%%%%%%%%%%%%%%%%%%%%%%%%%%%%%%%%%%%%%%%%%%%%%%%%%%%%%%%%%%%

\begin{document}
    \maketitle \thispagestyle{empty}


    %%%%%%%%%%%%%%%%%%%%%%%%%%%%%%%%%%%%%%%%%%%%%%%%%%%%%%%%%%%%%
    % Begin edit from here

    \maketitle \thispagestyle{empty}

    %		\begin{figure}[h]
    %			\centering
    %			\includegraphics[width=0.75\textwidth]{../HonorCode}
    %		\end{figure}
    %
    %		\begin{changemargin}{1 in}{1 in}
    %			We are giving you these solutions as review material to help you study
    %			for the rest of the course. Please do not share them with
    %			future students of CS 331 or post them on external sites.
    %		\end{changemargin}

    %		\newpage

    %%%%%%%%%%%%%%%%%%%%%%%%%%%%%%%%%%%%%%%%%%%%%%%%%%%%%%%%%%%%%
    % Begin edit from here
    \begin{homeworkProblem}[Problem 1]
        \maxpt{8}
        \begin{enumerate}
            \item \maxpt{6}\\
            A naive solution would be to split into 3 cases, one for each of the 3 possible
            operations\\
            (1) Adding a gap to the first string\\
            (2) Adding a gap to the second string\\
            (3) Including characters in both strings\\
            OPT(i, j) = min($\alpha_{x_i y_i} + OPT(i - 1, j - 1)$, $\delta + OPT(i - 1, j)$,
            $\delta + OPT(i, j - 1)$)\\
            This yields a time complexity of $O(3^{m + n})$\\
            Example: (CH, EN)\\
            First layer: (C, E), (C, \_), (\_, E)\\
            Second layer: (CH, EN), (CH, E\_), (C\_, EN), (CH, EN), (CH, \_\_), (C\_, \_E),
            (\_C, EN), (\_C, E\_), (\_\_, EN)\\
            % TODO
            \item \maxpt{2}
            \begin{itemize}
                \item [1.]
                \begin{tabular}{|c|c|c|c|c|}
                    \hline
                    \_ & A & L & G & O \\
                    \hline
                    T  & 1 & 2 & 3 & 4 \\
                    \hline
                    E  & 2 & 2 & 3 & 4 \\
                    \hline
                    S  & 3 & 3 & 3 & 4 \\
                    \hline
                    T  & 4 & 4 & 4 & 4 \\
                    \hline
                \end{tabular}\\
                $Alignment_1$: ALGO\\
                $Alignment_2$: TEST\\
                % TODO
                \item [2.]
                The cost is the value in the bottom right corner, 4\\
                Cost is 4 since there are 4 sets of characters that are different
                % TODO
            \end{itemize}
        \end{enumerate}
    \end{homeworkProblem}

    \begin{homeworkProblem}[Problem 2]
        \maxpt{12}
        \begin{enumerate}
            \item
            Indices (i, j) store the minimum number of cuts needed to make the substring from i
            to j a set of palindromes.\\
            \begin{lstlisting}
def min_palindrome(s):
    sections = [$\infty$] * len(s)
    for j in [1, len(s)]:
        min_sections = j
        for i in [1, j]:
            if isPali(i, j):
                min_sections = min(min_sections, sections[i - 1] + 1)
        sections[j] = min_sections
            \end{lstlisting}
            I'll test by running it on the string \textquoteleft coffee\textquoteright\\
            Each iteration runs with a fixed endpoint j and a increasing start point i.\\
            First iteration: ($1\to 1$, 1)
            \begin{tabular}{|c|c|c|c|c|c|c|}
                \hline
                $1$ & $\infty$ & $\infty$ & $\infty$ & $\infty$ & $\infty$ \\
                \hline
            \end{tabular}\\
            Second iteration: ($1\to 2$, 2)
            \begin{tabular}{|c|c|c|c|c|c|c|}
                \hline
                $1$ & $2$ & $\infty$ & $\infty$ & $\infty$ & $\infty$ \\
                \hline
            \end{tabular}\\
            Third iteration: ($1\to 3$, 3)
            \begin{tabular}{|c|c|c|c|c|c|c|}
                \hline
                $1$ & $2$ & $3$ & $\infty$ & $\infty$ & $\infty$ \\
                \hline
            \end{tabular}\\
            Fourth iteration: ($1\to 4$, 4)
            \begin{tabular}{|c|c|c|c|c|c|c|}
                \hline
                $1$ & $2$ & $3$ & $3$ & $\infty$ & $\infty$ \\
                \hline
            \end{tabular}\\
            Fifth iteration: ($1\to 5$, 5)
            \begin{tabular}{|c|c|c|c|c|c|c|}
                \hline
                $1$ & $2$ & $3$ & $3$ & $4$ & $\infty$ \\
                \hline
            \end{tabular}\\
            Sixth iteration: ($1\to 6$, 6)
            \begin{tabular}{|c|c|c|c|c|c|c|}
                \hline
                $1$ & $2$ & $3$ & $3$ & $4$ & $4$ \\
                \hline
            \end{tabular}\\
            % TODO
            \item
            % TODO
        \end{enumerate}
    \end{homeworkProblem}

    \begin{homeworkProblem}
        \maxpt{10}
        \begin{enumerate}
            \item
            Since we can't include the direct children of a manager, then we need to include
            subtrees 3 layers deep.\\
            This isn't greedy since local best choices don't always lead to the global best
            choice.\\
            % TODO
            \item
            I used memoization to store the maximum enjoyment for each person.\\
            \begin{lstlisting}
Map(person, enjoyment) = {}
def max_enjoyment(person):
    if person is None:
        return 0
    if person in Map:
        return Map(person)
    ce = 0
    for child in person.children:
        ce += max_enjoyment(child)

    gce = 0
    for child in person.children:
        for grandchild in child.children:
            gce += max_enjoyment(grandchild)
    Map(tree, max(ce, gce + person.enjoyment))
    return max(ce, gce + person.enjoyment)
            \end{lstlisting}
            Time complexity is $O(n)$\\
            % TODO
        \end{enumerate}
    \end{homeworkProblem}

\end{document}
