%! Author = Ian
%! Date = 3/17/2024

% Preamble
\documentclass{article}

% Packages
\usepackage{amsmath}
\usepackage{fancyhdr}
\usepackage[top=2cm, left=2cm, right=2cm, bottom=2cm]{geometry}
\usepackage{multicol}
\usepackage{amsfonts}
\usepackage{amssymb}
\setlength{\columnsep}{0.5cm}

\author{Ian Chen}
\date{\today}

% Header
\fancyhf{}
\fancyhead[L]{Ian Chen}
\fancyhead[C]{CS331 Exam 2 Notes}
\fancyhead[R]{\today}
\pagestyle{fancy}

% Document
\begin{document}
    \begin{multicols*}{2}
        \subsection*{Proof of Correctness}
        Total Correctness: Termination and Partial Correctness\\
        Partial Correctness: Loop invariants and induction\\
        Loop Invariant: A property that holds before and after each iteration of a loop\\
        Initialization: The loop invariant holds before the first iteration\\
        Maintenance: If the loop invariant holds before an iteration, it holds after the iteration\\
        Termination: When loop terminates, invariant gives useful property to show the algorithm is
        correct\\
        Iterative: Usually loop invariants\\
        Recursive: Usually induction
        \subsection*{Complexity}
        $a^{\log_b x} = x^{\log_b a}$\\
        $\log_b x = \frac{\log_c x}{\log_c b}$\\
        $\log_b M \cdot N = \log_b M + \log_b N$\\
        $\log_b \frac{M}{N} = \log_b M - \log_b N$\\
        $\log_b M^k = k\log_b M$\\
        Big Oh: $f(n)$ is $O(g(n))$ if $f(n) \leq cg(n)$ for $n \geq n_0 : c, n_0 > 0$\\
        Big Omega: $f(n)$ is $\Omega(g(n))$ if $f(n) \geq cg(n)$ for $n \geq n_0 : c, n_0 > 0$\\
        Big Theta: $f(n)$ is $\Theta(g(n))$ if $f(n)$ is $O(g(n))$ and $f(n)$ is $\Omega(g(n))$\\
        Little Oh: Strict Big Oh\\
        Little Omega: Strict Big Omega\\
        $\lim_{n\to\infty}\frac{f(n)}{g(n)}$:\\
        0 if $f(n)$ is $o(g(n))$, $\infty$ if $f(n)$ is $\omega(g(n))$\\
        $< \infty$ if $f(n)$ is $O(g(n))$, $> 0$ if $f(n)$ is $\Omega(g(n))$\\
        $0 < \infty$ if $f(n)$ is $\Theta(g(n))$\\
        Growth Rates: $1 < \log(n) < \sqrt{n} < n < n\log(n) < n^2 < n^c < 2^n < c^n < n! < n^n$\\
        Harmonic: $\sum_{k=1}^{n} \frac{1}{k} \sim \ln n$\\
        Triangular: $\sum_{k=1}^{n} k = \frac{n(n + 1)}{2} \sim \frac{n^2}{2}$\\
        Squares: $\sum_{k=1}^{n} k^2 \sim \frac{n^3}{3}$\\
        Geometric: $\sum_{k=0}^{n} ar^k = \frac{a(r^{n+1} - 1)}{r - 1}$\\
        Stirling's Approximation: $\log_2(n!) \sim n\log_2 n$\\
        Master's Theorem: $T(n) = aT(\frac{n}{b}) + f(n)$, $\epsilon > 0$\\
        $f(n) = O(n^{\log_b (a) - \epsilon}) \to T(n) = \Theta(n^{\log_b a})$\\
        $f(n) = \Theta(n^{\log_b a}) \to T(n) = \Theta(n^{\log_b a}\log n)$\\
        $f(n) = \Omega(n^{log_b (a) + \epsilon}) \wedge af(\frac{n}{b}) \leq cf(n)$ for some $c
        < 1 \to T(n) = \Theta(f(n))$
        \subsection*{Divide and Conquer}
        Divide: Break problem into smaller subproblems\\
        Conquer/Combine: Solve subproblems recursively and combine
        \subsubsection*{Recurrence Relation}
        A function defined in terms of itself\\
        Mirrors the recursive algorithm it represents\\
        Analysis of the running time of a divide and conquer algorithm generally involves
        solving a recurrence relation\\
        1,2,3 Method
        \subsubsection*{Merge Sort}
        $T(n) = 2T(\frac{n}{2}) + n-1$, $T(1) = 0$\\
        \begin{tabular}{c|c|c}
            Level  & Problem Size        & Total Time               \\
            \hline
            0      & $n$                 & $n$                      \\
            1      & $\frac{n}{2}$       & $2\frac{n}{2} = n$       \\
            2      & $\frac{n}{4}$       & $4\frac{n}{4} = n$       \\
            \vdots & \vdots              & \vdots                   \\
            $k$    & $\frac{n}{2^k} = 1$ & $2^{k}\frac{n}{2^k} = n$ \\
        \end{tabular}\\
        $\to (\sum\limits_{i=0}^{k-1} n) + 0 \cdot 2^{log_2 n} \to \sum\limits_{i=0}^{log_2 n -
        1} n \to n\log_2 n$\\
        or $\to \sum\limits_{i=0}^{k} n \to \sum\limits_{i=0}^{log_2 n} n \to n\log_2 n$
        \subsubsection*{Closest Pair}
        \subsection*{Dynamic Programming}
        \subsubsection*{Weighted Interval Scheduling}
        \subsubsection*{Memoization}
        \subsubsection*{Subset Sum/Knapsacks}
        \subsubsection*{Sequence Alignment}
        \subsubsection*{Bellman-Ford}
        \subsection*{Network Flow}
        \subsubsection*{Maximum Flow Problem}
        \subsubsection*{Ford-Fulkerson}
        \subsubsection*{Max Flow/Min Cut}
    \end{multicols*}
\end{document}