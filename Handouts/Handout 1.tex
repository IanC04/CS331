%! Author = Ian
%! Date = 1/18/2024

% Preamble
\documentclass[11pt]{article}

% Packages
\usepackage{amsmath}
\usepackage{amsthm}
\usepackage{amsfonts}
\usepackage{amssymb}

\title{Handout 1}
\author{Ian Chen}
\date{\today}

% Document
\begin{document}

    \maketitle

    \setcounter{section}{0} % One less than the book's section number.


    \section{Problem 1}\label{sec:chapter_1}

    \newtheorem{theorem}{Theorem}
    \newtheorem{exercise}[theorem]{Exercise}
    \newtheorem{corollary}[theorem]{Corollary}
    \newtheorem{question}[theorem]{Question}
    \newtheorem{lemma}[theorem]{Lemma}

    \setcounter{theorem}{0} % One less than the first theorem number

    \begin{theorem}
        Prove that for any tree, the number of edges is one less than the number of nodes, i.e., n
        - m = 1.
    \end{theorem}
    \begin{proof}
        % TODO Finish Proof
    \end{proof}


    \section{Problem 2}\label{sec:chapter_2}

    \begin{theorem}
        Prove that such an algorithm cannot possibly exist.
    \end{theorem}
    \begin{proof}
        % TODO Finish Proof
    \end{proof}


    \section{Problem 3}\label{sec:chapter_3}

    \begin{theorem}
        Prove that a bishop placed on that square can go to any black colored square on the
        chessboard.
    \end{theorem}
    \begin{proof}
        % TODO Finish Proof
    \end{proof}


    \section{Problem 4}\label{sec:chapter_4}

    \begin{theorem}
        Prove that this new board cannot be tiled with dominoes$\textemdash$that is, any attempt to
        cover the chessboard with dominoes must always have either an uncovered square or a
        domino hanging off the edge.
    \end{theorem}
    \begin{proof}
        % TODO Finish Proof
    \end{proof}
\end{document}