\documentclass[12pt]{article}
\usepackage{amsmath,amsfonts,amsthm,amssymb}
\usepackage{setspace}
\usepackage{fancyhdr}
\usepackage{lastpage}
\usepackage{extramarks}
\usepackage[ruled,vlined]{algorithm2e}
\usepackage{chngpage}
\usepackage{soul,color}
\usepackage{graphicx,float,wrapfig}
\usepackage{ listings}
\usepackage{booktabs}
\usepackage[shortlabels]{enumitem}
\usepackage{array}% http://ctan.org/pkg/array

\newcolumntype{M}{>{\centering\arraybackslash}m{\dimexpr.25\linewidth-2\tabcolsep}}
\newcommand{\Class}{ \normalsize CS 331: Algorithms and Complexity (Fall 2023)\\
\small    Unique Number: 52765, 52770
}


%\newcommand{\ClassInstructor}{Fares}
% Homework Specific Information. Change it to your own
\newcommand{\Title}{Assignment 4 - Solution}
\newcommand{\DueDate}{Tuesday, 27 February, by 11.59pm}

\newcommand{\StudentName}{}
\newcommand{\StudentClass}{}
\newcommand{\StudentNumber}{}

% In case you need to adjust margins:
\topmargin=-0.45in      %
\evensidemargin=0in     %
\oddsidemargin=0in      %
\textwidth=6.5in        %
\textheight=9.0in       %
\headsep=0.25in         %

% Setup the header and footer
\pagestyle{fancy}                                                       %
\lhead{\StudentName}                                                 %
\chead{\Title}  %
\rhead{\firstxmark}                                                     %
\lfoot{\lastxmark}                                                      %
\cfoot{}                                                                %
\rfoot{Page\ \thepage\ of\ \protect\pageref{LastPage}}                          %
\renewcommand\headrulewidth{0.4pt}                                      %
\renewcommand\footrulewidth{0.4pt}                                      %

%%%%%%%%%%%%%%%%%%%%%%%%%%%%%%%%%%%%%%%%%%%%%%%%%%%%%%%%%%%%%
% Some tools
\newcommand{\enterProblemHeader}[1]{\nobreak\extramarks{#1}{#1 continued on next page\ldots}\nobreak%
\nobreak\extramarks{#1 (continued)}{#1 continued on next page\ldots}\nobreak}%
\newcommand{\exitProblemHeader}[1]{\nobreak\extramarks{#1 (continued)}{#1 continued on next page
\ldots}\nobreak%
\nobreak\extramarks{#1}{}\nobreak}%

\newcommand{\homeworkProblemName}{}%
\newcounter{homeworkProblemCounter}%
\newenvironment{homeworkProblem}[1][Problem \arabic{homeworkProblemCounter}]%
{\stepcounter{homeworkProblemCounter}%
\renewcommand{\homeworkProblemName}{#1}%
\section*{\homeworkProblemName}%
\enterProblemHeader{\homeworkProblemName}}%
{\exitProblemHeader{\homeworkProblemName}}%

\newcommand{\homeworkSectionName}{}%
\newlength{\homeworkSectionLabelLength}{}%
\newenvironment{homeworkSection}[1]%
{% We put this space here to make sure we're not connected to the above.

    \renewcommand{\homeworkSectionName}{#1}%
    \settowidth{\homeworkSectionLabelLength}{\homeworkSectionName}%
    \addtolength{\homeworkSectionLabelLength}{0.25in}%
    \changetext{}{-\homeworkSectionLabelLength}{}{}{}%
    \subsection*{\homeworkSectionName}%
    \enterProblemHeader{\homeworkProblemName\ [\homeworkSectionName]}}%
    {\enterProblemHeader{\homeworkProblemName}%

    % We put the blank space above in order to make sure this margin
    % change doesn't happen too soon.
    \changetext{}{+\homeworkSectionLabelLength}{}{}{}}%

\newcommand{\Answer}{\ \\\textbf{Answer:} }
\newcommand{\Acknowledgement}[1]{\ \\{\bf Acknowledgement:} #1}

%%%%%%%%%%%%%%%%%%%%%%%%%%%%%%%%%%%%%%%%%%%%%%%%%%%%%%%%%%%%%


%%%%%%%%%%%%%%%%%%%%%%%%%%%%%%%%%%%%%%%%%%%%%%%%%%%%%%%%%%%%%
% Make title
\title{\textmd{\bf \Class\\ \Title}\\\vspace{0.1in}\small{Due\ on\ \DueDate}}
\date{}
\author{\textbf{\StudentName}\ \ \StudentClass\ \ \StudentNumber}
%%%%%%%%%%%%%%%%%%%%%%%%%%%%%%%%%%%%%%%%%%%%%%%%%%%%%%%%%%%%%
\setlist[enumerate,1]{label=\bfseries(\alph*)}
\newcommand{\maxpt}[1]{\ifthenelse{\equal{#1}{1}}{\textbf{(#1 pt)}}{\textbf{(#1 pts)}}}

\begin{document}
    \maketitle \thispagestyle{empty}

    %\begin{figure}[h]
    %	\centering
    %	\includegraphics[width=0.75\textwidth]{../HonorCode}
    %\end{figure}
    %		\newpage


    %%%%%%%%%%%%%%%%%%%%%%%%%%%%%%%%%%%%%%%%%%%%%%%%%%%%%%%%%%%%%
    % Begin edit from here
    \begin{homeworkProblem}
        \textbf{(10 points)}
        \begin{enumerate}
            \item \textbf{(1pt each)}\\
            $T_2(n)$ has $\frac{4}{3}$ inside the recurrence, while results in subsequent calls
            growing the value of n.\\
            $T_3(n)$ has $- 5n^3$ as the cost, which results in negative time complexity.
            \item \textbf{(2pt each)}\\
            \begin{itemize}
                \item $T_1(n) = 2T_1(\frac{n}{4}) + n^2$, $T_1(1) = 1$\\
                Using Master's Theorem, a = 2, b = 4, f(n) = $n^2$\\
                $n^{\log_b a} = n^{\log_4 2} = n^{\frac{1}{2}}$\\
                $f(n) = \Omega(n^{\log_b a})$, so $T_1(n) = \Theta(f(n)) = \Theta(n^2)$\\
                \item $T_4(n) = 2T_4(\frac{n}{2}) + n\log n$, $T_4(2) = 0$\\
                Using Master's Theorem, a = 2, b = 2, f(n) = $n\log n$\\
                $n^{\log_b a} = n^{\log_2 2} = n$\\
                However, f(n) = $n\log n$ is not polynomially larger than $n$, so we cannot use
                Master's Theorem.\\
                I will use the recurrence relation\\
                \begin{tabular}{c|c|c}
                    Level & Problem size & Total time\\
                    \hline
                    0 & n & $n\log n$\\
                    1 & $\frac{n}{2}$ & $2 \cdot \frac{n}{2}\log \frac{n}{2}$\\
                    2     & $\frac{n}{4}$ & $2^2 \cdot \frac{n}{2^2}\log \frac{n}{2^2}$ \\
                    \vdots & \vdots & \vdots\\
                    k & $\frac{n}{2^k} = 1$ & $2^k \cdot \frac{n}{2^k}\log \frac{n}{2^k}$\\
                \end{tabular}\\
                $\to (\sum\limits_{i=0}^{k-1} 2^i \cdot \frac{n}{2^i}\log \frac{n}{2^i}) + 1 \cdot
                2^k \to (\sum\limits_{i=0}^{k-1} n\log \frac{n}{2^i}) + 2^k
                \to (\sum\limits_{i=0}^{log_2(n)-1} n\log \frac{n}{2^i}) + 2^{log_2{n}}
                \to (\sum\limits_{i=0}^{log_2(n)-1} n\log n - n\log 2^i) + n
                \to n(\sum\limits_{i=0}^{log_2(n)-1} \log n - i) + n$\\
                % TODO
                \item $T_5(n) = \sqrt{n}T_5(\sqrt{n}) + n$, $n \geq 2$\\
                We cannot use Master's Theorem here, as the form of the recurrence is not in the
                form of $T(n) = aT(\frac{n}{b}) + f(n)$.\\
                % TODO
            \end{itemize}
            \item \maxpt{2}
            % TODO
        \end{enumerate}
    \end{homeworkProblem}

    \newpage

    \begin{homeworkProblem}
        \textbf{(10 points)}
        \begin{enumerate}
            \item (4 points)
            % TODO
            \item (2 points)
            % TODO
            \item (2 points)
            % TODO
            \item (2 points)
            % TODO
        \end{enumerate}
    \end{homeworkProblem}

    \newpage

    \begin{homeworkProblem}
        \maxpt{10}

        \begin{enumerate}
            \item
            % TODO
            \item
            % TODO
            \item
            % TODO
        \end{enumerate}
    \end{homeworkProblem}
\end{document}
%%% Local Variables:
%%% mode: latex
%%% TeX-master: t
%%% End:
