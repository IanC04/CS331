\documentclass[12pt]{article}
\usepackage{amsmath,amsfonts,amsthm,amssymb}
\usepackage{setspace}
\usepackage{fancyhdr}
\usepackage{lastpage}
\usepackage{extramarks}
\usepackage{hyperref}
\hypersetup{
    urlcolor=blue,
}
\usepackage[ruled,vlined]{algorithm2e}
\usepackage{chngpage}
\usepackage{soul,color}
\usepackage{graphicx,float,wrapfig}
\usepackage{ listings}
\usepackage{comment}
\usepackage{arabicfnt}
\newcommand{\Class}{ \normalsize CS 331: Algorithms and Complexity (Spring 2024)\\
\small    Unique Number: 50930, 50935 50940, 50945
}
%\newcommand{\ClassInstructor}{Fares}
% Homework Specific Information. Change it to your own
\newcommand{\Title}{Assignment 1}
\newcommand{\DueDate}{Thursday, 25 January, by 11.59pm}
\newcommand{\StudentName}{}
\newcommand{\StudentClass}{}
\newcommand{\StudentNumber}{}


\setlength\parindent{0pt}
% In case you need to adjust margins:
\topmargin=-0.45in      %
\evensidemargin=0in     %
\oddsidemargin=0in      %
\textwidth=6.5in        %
\textheight=9.0in       %
\headsep=0.25in         %

% Setup the header and footer
\pagestyle{fancy}                                                       %
\lhead{\StudentName}                                                 %
\chead{\Title}  %
\rhead{\firstxmark}                                                     %
\lfoot{\lastxmark}                                                      %
\cfoot{}                                                                %
\rfoot{Page\ \thepage\ of\ \protect\pageref{LastPage}}                          %
\renewcommand\headrulewidth{0.4pt}                                      %
\renewcommand\footrulewidth{0.4pt}                                      %

%%%%%%%%%%%%%%%%%%%%%%%%%%%%%%%%%%%%%%%%%%%%%%%%%%%%%%%%%%%%%
% Some tools
\newcommand{\enterProblemHeader}[1]{\nobreak\extramarks{#1}{#1 continued on next page\ldots}\nobreak%
\nobreak\extramarks{#1 (continued)}{#1 continued on next page\ldots}\nobreak}%
\newcommand{\exitProblemHeader}[1]{\nobreak\extramarks{#1 (continued)}{#1 continued on next page
\ldots}\nobreak%
\nobreak\extramarks{#1}{}\nobreak}%

\newcommand{\homeworkProblemName}{}%
\newcounter{homeworkProblemCounter}%
\newenvironment{homeworkProblem}[1][Problem \arabic{homeworkProblemCounter}]%
{\stepcounter{homeworkProblemCounter}%
\renewcommand{\homeworkProblemName}{#1}%
\section*{\homeworkProblemName}%
\enterProblemHeader{\homeworkProblemName}}%
{\exitProblemHeader{\homeworkProblemName}}%

\newcommand{\homeworkSectionName}{}%
\newlength{\homeworkSectionLabelLength}{}%
\newenvironment{homeworkSection}[1]%
{% We put this space here to make sure we're not connected to the above.

    \renewcommand{\homeworkSectionName}{#1}%
    \settowidth{\homeworkSectionLabelLength}{\homeworkSectionName}%
    \addtolength{\homeworkSectionLabelLength}{0.25in}%
    \changetext{}{-\homeworkSectionLabelLength}{}{}{}%
    \subsection*{\homeworkSectionName}%
    \enterProblemHeader{\homeworkProblemName\ [\homeworkSectionName]}}%
    {\enterProblemHeader{\homeworkProblemName}%

    % We put the blank space above in order to make sure this margin
    % change doesn't happen too soon.
    \changetext{}{+\homeworkSectionLabelLength}{}{}{}}%

\newcommand{\Answer}{\ \\\textbf{Answer:} }
\newcommand{\Acknowledgement}[1]{\ \\{\bf Acknowledgement:} #1}


\def\changemargin#1#2{\list{}{\rightmargin#2\leftmargin#1}\item[]}
\let\endchangemargin=\endlist

%%%%%%%%%%%%%%%%%%%%%%%%%%%%%%%%%%%%%%%%%%%%%%%%%%%%%%%%%%%%%


%%%%%%%%%%%%%%%%%%%%%%%%%%%%%%%%%%%%%%%%%%%%%%%%%%%%%%%%%%%%%
% Make title
\title{\textmd{\bf \Class\\ \Title}\\\vspace{0.1in}\small{Due\ on\ \DueDate}}
\date{}
\author{\textbf{\StudentName}\ \ \StudentClass\ \ \StudentNumber}
%%%%%%%%%%%%%%%%%%%%%%%%%%%%%%%%%%%%%%%%%%%%%%%%%%%%%%%%%%%%%

\begin{document}
    \maketitle \thispagestyle{empty}

    %%%%%%%%%%%%%%%%%%%%%%%%%%%%%%%%%%%%%%%%%%%%%%%%%%%%%%%%%%%%%
    % Begin edit from here

    \begin{homeworkProblem}
        \textbf{(a) (3 points)}\newline
        \texttt{O(f$_1$(n)) = n}\newline
        \texttt{O(f$_2$(n)) = n$\log\log$n}\newline
        \texttt{O(f$_3$(n)) = n$^2$}\newline
        \textit{We need to find whether $\sqrt {n}$ or $\log(n^2)$ grows faster.\newline
        Solve: $\lim_{n\to\infty}\frac{\sqrt{n}}{\log(n^2)}$ =
            $\lim_{n\to\infty}\frac{d}{dn}\frac{\sqrt{n}}{\log(
            n^2)}$ = $\lim_{n\to\infty}\frac{1}{2\sqrt{n}} / \frac{2n}{n^2}$ (Constants don't
            matter), this approaches $\infty$ as n increases, $\therefore$}\newline
        \texttt{O(f$_4$(n)) = $\sqrt{n}$}\newline
        \texttt{O(f$_5$(n)) = $2^{\sqrt{n}}$}\newline
        \texttt{O(f$_6$(n)) = $2^{\log n}$}\newline
        \textit{$\therefore$ the order is:}
        \texttt{$f_{4}(n), f_{1}(n), f_{2}(n), f_{3}(n), f_{6}(n), f_{5}(n)$}\newline

        \textbf{(b) (3 points)}\newline
        \textit{Given a$_1$ = 1, a$_2$ = 8, and a$_n$ = a$_{n-1}$ + 2a$_{n-2}$ when n $\geq$ 3,
            show a$_n$ = 3 $\cdot$ 2$^{n-1}$ + 2(-1)$^{n}$ for all n $\in$ $\mathbb{N}$.}\newline
        \texttt{Define our predicate as P(n) = a$_n$ = 3 $\cdot$ 2$^{n-1}$ + 2(-1)$^{n}$}\newline
        \textbf{Base Cases:}
        \begin{proof}
            Let n = 1, then a$_1$ = 1, and 3 $\cdot$ 2$^{1-1}$ + 2(-1)$^{1}$ = 1, which is
            valid, $\therefore$ P(1) is true.\newline
            Let n = 2, then a$_2$ = 8, and 3 $\cdot$ 2$^{2-1}$ + 2(-1)$^{2}$ = 8, which is
            valid, $\therefore$ P(2) is true.
        \end{proof}

        \textbf{Inductive Hypothesis:}
        \textit{Assume for n = k, P(1), P(2), $\ldots$, P(k) are all true.}\newline

        \textbf{Inductive Step:}\newline
        \textit{Show P(k + 1), aka a$_{k+1}$ = 3 $\cdot$ 2$^{k}$ + 2(-1)$^{k+1}$.}
        \begin{proof}
            We are given this equation: a$_{k+1}$ = a$_{k}$ + 2a$_{k-1}$.\newline
            We can substitute our inductive hypothesis into this equation to get:
            a$_{k+1}$ = (3 $\cdot$ 2$^{k-1}$ + 2(-1)$^{k}$) + 2(3 $\cdot$ 2$^{k-2}$ + 2(-1)$^{k-1}$).\newline
            We can factor out a two from the first group to get 2(3 $\cdot$ 2$^{k-2}$ + 2(-1)$^{
                k-1}$) + 2(3 $\cdot$ 2$^{k-2}$ + 2(-1)$^{k-1}$).\newline
            Combine the two groups to get 4(3 $\cdot$ 2$^{k-2}$ + 2(-1)$^{k-1}$).\newline
            Since 4 = 2$^2$, we can add 2 to each exponent to get 3 $\cdot$ 2$^{k}$ + 2(-1)$^{k+1}$.\newline
            $\therefore$, we have shown that P(k + 1) is true, and by induction, P(n) is true for
            all n $\in$ $\mathbb{N}$
        \end{proof}

        \textbf{(c)} \textbf{(4 points)}\newline
        \texttt{Algorithm 1(Alice)}
        \begin{proof}
            We need to swap the lines \texttt{Report average as sum / count;} and \texttt{
                Increase count by 1;} as in the first iteration, we will divide by 0, which is
            not the correct average.\newline
        \end{proof}
        \texttt{Algorithm 2(Bob)}
        \begin{proof}
            First, we show termination.\newline
            Since each iteration we read one integer, the set of integers needed to be read is
            decreased by one each iteration, and since the set is finite, the algorithm will
            terminate.\newline
            Next, we show correctness.\newline
            Assume we have to read n integers from the input stream, which we'll call S.\newline
            \textbf{Base Case: }n = 1\newline
            average is updated to $\frac{(0 \cdot 0 + S_1)}{0 + 1}$ = S$_1$, which is the correct
            average, and count is incremented by one which results in count=1.\newline
            Then we have average=S$_1$ and count=1, which is correct.\newline
            \textbf{Inductive Hypothesis: }Assume for n = k, average=$\frac{\sum_{i=1}^{k}S_i}{
                count}$
            and count=k.\newline
            \textbf{Inductive Step: }Show for n = k + 1, average=$\frac{\sum_{i=1}^{k+1}S_i}{
                count + 1}$ and count=k + 1.\newline
            In the loop, we read the integer S$_{k+1}$ and update average to $\frac{(average*count
                )+S_{k+1}}{
                count + 1}$.\newline
            We have to show that $\sum_{i=1}^{k+1}S_i$ = (average * count) + S$_{k+1}$.\newline
            We know that average=$\frac{\sum_{i=1}^{k}S_i}{
                count}$, $\therefore$ average * count = $\sum_{i=1}^{k}S_i$.\newline
            Adding S$_{k+1}$, we get $\sum_{i=1}^{k+1}S_i$ = (average * count) + S$_{k+1}$.\newline
            $\therefore$, we can express $\frac{(average*count)+S_{k+1}}{count + 1}$ as $\frac{
                \sum_{i=1}^{k+1}S_i}{count + 1}$, which is the correct average.\newline
            count is then incremented by one, which results in count=k+1.\newline
            $\therefore$, we have shown that the algorithm is correct.\newline
        \end{proof}


    \end{homeworkProblem}

    \begin{homeworkProblem}
        \textbf{(6 points)}\newline
        \textit{For this problem, I would use a modified binary search, since I know that the
        array is sorted in ascending order with the target being less than the "max" values
        appended to the array.}
        \begin{verbatim}
// Finding bounds
int low = 0, high = 1;
while(E[high] < x) {
    low = high;
    high *= 2;
}
while (low <= high) {
    int mid = low + (high - low) / 2; // or (low + high) >>> 1
    if (E[mid] == x) return mid;
    else if (E[mid] < x) low = mid + 1;
    else high = mid - 1;
}
        \end{verbatim}
        \begin{proof}
            This algorithm terminates since the range of the search is halved each iteration.\newline
            Correctness: We are trying to find x $<$ max in E.\newline
            Since the array is constructed as \textbf{[sorted$\mid$max$_{size-n}$]}, and max is
            greater than all elements of the array up to n, then there are 3 cases in each ith
            iteration:\newline
            \textbf{Case 1:} x $<$ E[i], then since the array is sorted, x is the lower half of
            the search space.\newline
            \textbf{Case 2:} x $>$ E[i], then since the array is sorted, x is the upper half of
            the search space.\newline
            \textbf{Case 3:} x = E[i], then we have found the index of x.\newline
            $\therefore$, we half the search space each iteration, thus the algorithm is O($\log
            n$).
        \end{proof}

    \end{homeworkProblem}


    \begin{homeworkProblem}
        Your task is to do the following:
        \begin{enumerate}
            \item[i] \textbf{(7 points)}
            \begin{verbatim}
Initialize all TAs to be unassigned
Store all TA preferences in a sorted list for each course and the
number of TAs needed
while (course needs TA) {
    c = a course that needs a TA
    ta = first TA in c's preference list that c hasn't tried to assign
    if (ta is unqualified for c) {
        c rejects ta
    } else if (ta is unassigned) {
        assign ta to c
    } else if (ta prefers c to ta's current course) {
        c' = ta's current course
        unassign ta from c'
        assign ta to c
    }
    else {
        ta rejects c
    }
}
            \end{verbatim}
            \item[ii] \textbf{(7 points)}
            \begin{proof}
                First, we show termination.\newline
                The sorted list of applicants TAs for each course is strictly decreasing each
                iteration, $\therefore$ the algorithm will terminate.\newline
                Next, we show correctness.\newline
                Observation 1: Courses are assigned TAs in order of preference.\newline
                Observation 2: Once a TA is assigned to a course, they will not be unassigned,
                only reassigned to a different course.\newline
                Claim 1: All available TA spots are filled unless unqualified.\newline
                Proof: Assume by contradiction that there is an available TA spot upon termination,
                in other words, there is a course c$_0$ that needs a TA and there is one that's
                qualified and unassigned.\newline
                Since there are more TAs than courses, there must be a TA t$_0$ that is unassigned
                .\newline
                Since t$_0$ is unassigned, then c$_0$ must have rejected t$_0$, however, since t$
                _0$ is unassigned, then t$_0$ must be unqualified for c$_0$.\newline
                $\therefore$ we have a contradiction, and all available TA spots are filled
                unless unqualified.\newline
                Claim 2: All TAs are assigned to a course in a stable marriage.\newline
                Proof: Assume by contradiction that there is a TA t$_0$ that is assigned to
                course c$_1$ and TA t$_1$ that is assigned to course c$_0$ such that c$_0$
                prefers t$_0$ over t$_1$ and t$_0$ prefers c$_0$ over c$_1$.\newline
                Case 1: t$_0$ was never assigned to c$_0$.\newline
                Since c$_0$ prefers t$_0$ over t$_1$ and c$_0$ is assigned TAs based on preference,
                then t$_0$ must have been assigned to a course that is higher on t$_0$\'s
                preference list than c$_0$, which is a contradiction.\newline
                Case 2: t$_0$ was assigned to c$_0$ and then reassigned.\newline
                This means that t$_0$ rejected c$_0$ and was assigned to a course that is higher
                on their preference list than c$_0$.\newline
                However, since c$_1$ is lower on t$_0$\'s preference list than c$_0$, this is a
                contradiction.\newline
                $\therefore$, we have shown that all TAs are assigned to a course in a stable
                marriage.\newline
                $\therefore$, we have shown that the algorithm is totally correct.\newline
            \end{proof}
        \end{enumerate}


    \end{homeworkProblem}
\end{document}

 