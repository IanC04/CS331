\documentclass[12pt]{article}
\usepackage{amsmath,amsfonts,amsthm,amssymb}
\usepackage{setspace}
\usepackage{fancyhdr}
\usepackage{lastpage}
\usepackage{extramarks}
\usepackage[ruled,vlined]{algorithm2e}
\usepackage{chngpage}
\usepackage{soul,color}
\usepackage{graphicx,float,wrapfig}
\usepackage{listings}
\lstset{
    basicstyle=\ttfamily,
    mathescape
}
\usepackage{booktabs}
\usepackage[shortlabels]{enumitem}
\usepackage{array}% http://ctan.org/pkg/array

\newcolumntype{M}{>{\centering\arraybackslash}m{\dimexpr.25\linewidth-2\tabcolsep}}
\newcommand{\Class}{ \normalsize CS 331: Algorithms and Complexity (Fall 2023)\\
\small    Unique Number: 52765, 52770
}


%\newcommand{\ClassInstructor}{Fares}
% Homework Specific Information. Change it to your own
\newcommand{\Title}{Assignment 4 - Solution}
\newcommand{\DueDate}{Tuesday, 27 February, by 11.59pm}

\newcommand{\StudentName}{}
\newcommand{\StudentClass}{}
\newcommand{\StudentNumber}{}

% In case you need to adjust margins:
\topmargin=-0.45in      %
\evensidemargin=0in     %
\oddsidemargin=0in      %
\textwidth=6.5in        %
\textheight=9.0in       %
\headsep=0.25in         %

% Setup the header and footer
\pagestyle{fancy}                                                       %
\lhead{\StudentName}                                                 %
\chead{\Title}  %
\rhead{\firstxmark}                                                     %
\lfoot{\lastxmark}                                                      %
\cfoot{}                                                                %
\rfoot{Page\ \thepage\ of\ \protect\pageref{LastPage}}                          %
\renewcommand\headrulewidth{0.4pt}                                      %
\renewcommand\footrulewidth{0.4pt}                                      %

%%%%%%%%%%%%%%%%%%%%%%%%%%%%%%%%%%%%%%%%%%%%%%%%%%%%%%%%%%%%%
% Some tools
\newcommand{\enterProblemHeader}[1]{\nobreak\extramarks{#1}{#1 continued on next page\ldots}\nobreak%
\nobreak\extramarks{#1 (continued)}{#1 continued on next page\ldots}\nobreak}%
\newcommand{\exitProblemHeader}[1]{\nobreak\extramarks{#1 (continued)}{#1 continued on next page
\ldots}\nobreak%
\nobreak\extramarks{#1}{}\nobreak}%

\newcommand{\homeworkProblemName}{}%
\newcounter{homeworkProblemCounter}%
\newenvironment{homeworkProblem}[1][Problem \arabic{homeworkProblemCounter}]%
{\stepcounter{homeworkProblemCounter}%
\renewcommand{\homeworkProblemName}{#1}%
\section*{\homeworkProblemName}%
\enterProblemHeader{\homeworkProblemName}}%
{\exitProblemHeader{\homeworkProblemName}}%

\newcommand{\homeworkSectionName}{}%
\newlength{\homeworkSectionLabelLength}{}%
\newenvironment{homeworkSection}[1]%
{% We put this space here to make sure we're not connected to the above.

    \renewcommand{\homeworkSectionName}{#1}%
    \settowidth{\homeworkSectionLabelLength}{\homeworkSectionName}%
    \addtolength{\homeworkSectionLabelLength}{0.25in}%
    \changetext{}{-\homeworkSectionLabelLength}{}{}{}%
    \subsection*{\homeworkSectionName}%
    \enterProblemHeader{\homeworkProblemName\ [\homeworkSectionName]}}%
    {\enterProblemHeader{\homeworkProblemName}%

    % We put the blank space above in order to make sure this margin
    % change doesn't happen too soon.
    \changetext{}{+\homeworkSectionLabelLength}{}{}{}}%

\newcommand{\Answer}{\ \\\textbf{Answer:} }
\newcommand{\Acknowledgement}[1]{\ \\{\bf Acknowledgement:} #1}

%%%%%%%%%%%%%%%%%%%%%%%%%%%%%%%%%%%%%%%%%%%%%%%%%%%%%%%%%%%%%


%%%%%%%%%%%%%%%%%%%%%%%%%%%%%%%%%%%%%%%%%%%%%%%%%%%%%%%%%%%%%
% Make title
\title{\textmd{\bf \Class\\ \Title}\\\vspace{0.1in}\small{Due\ on\ \DueDate}}
\date{}
\author{\textbf{\StudentName}\ \ \StudentClass\ \ \StudentNumber}
%%%%%%%%%%%%%%%%%%%%%%%%%%%%%%%%%%%%%%%%%%%%%%%%%%%%%%%%%%%%%
\setlist[enumerate,1]{label=\bfseries(\alph*)}
\newcommand{\maxpt}[1]{\ifthenelse{\equal{#1}{1}}{\textbf{(#1 pt)}}{\textbf{(#1 pts)}}}

\begin{document}
    \maketitle \thispagestyle{empty}

    %\begin{figure}[h]
    %	\centering
    %	\includegraphics[width=0.75\textwidth]{../HonorCode}
    %\end{figure}
    %		\newpage


    %%%%%%%%%%%%%%%%%%%%%%%%%%%%%%%%%%%%%%%%%%%%%%%%%%%%%%%%%%%%%
    % Begin edit from here
    \begin{homeworkProblem}
        \textbf{(10 points)}
        \begin{enumerate}
            \item \textbf{(1pt each)}\\
            $T_2(n)$ has $\frac{4}{3}$ inside the recurrence, while results in subsequent calls
            growing the value of n.\\
            $T_3(n)$ has $- 5n^3$ as the cost, which results in negative time complexity.
            \item \textbf{(2pt each)}
            \begin{itemize}
                \item $T_1(n) = 2T_1(\frac{n}{4}) + n^2$, $T_1(1) = 1$\\
                Using Master's Theorem, a = 2, b = 4, f(n) = $n^2$\\
                $n^{\log_b a} = n^{\log_4 2} = n^{\frac{1}{2}}$\\
                $f(n) = \Omega(n^{\log_b a})$, so $T_1(n) = \Theta(f(n)) = \Theta(n^2)$\\
                \item $T_4(n) = 2T_4(\frac{n}{2}) + n\log n$, $T_4(2) = 0$\\
                Using Master's Theorem, a = 2, b = 2, f(n) = $n\log n$\\
                $n^{\log_b a} = n^{\log_2 2} = n$\\
                However, f(n) = $n\log n$ is not polynomially larger than $n$, so we cannot use
                Master's Theorem.\\
                I will use the recurrence relation\\
                \begin{tabular}{c|c|c}
                    Level  & Problem size        & Total time                                  \\
                    \hline
                    0      & n                   & $n\log n$                                   \\
                    1      & $\frac{n}{2}$       & $2 \cdot \frac{n}{2}\log \frac{n}{2}$       \\
                    2      & $\frac{n}{2^2}$     & $2^2 \cdot \frac{n}{2^2}\log \frac{n}{2^2}$ \\
                    \vdots & \vdots              & \vdots                                      \\
                    k      & $\frac{n}{2^k} = 2$ & $2^k \cdot \frac{n}{2^k}\log \frac{n}{2^k}$ \\
                \end{tabular}\\
                First, I'll solve for k\\
                $\frac{n}{2^k} = 2$\\
                $\to n = 2^{k+1}$\\
                $\to \log_2 n = k + 1$\\
                $\to k = \log_2 n - 1$\\
                Then, I'll sum the total time\\
                $\to \sum\limits_{i=0}^{k-1} 2^i \cdot \frac{n}{2^i}\log \frac{n}{2^i} + 0 \cdot
                2^k$\\
                $\to \sum\limits_{i=0}^{\log_2 n - 2} n\log \frac{n}{2^i}$\\
                $\to \sum\limits_{i=0}^{\log_2 n - 2} n\log n - n\log 2^i$\\
                $\to n\log n(\log n - 2) - \sum\limits_{i=0}^{\log_2 n - 2} ni$\\
                $\to n\log n(\log n - 2) - n\frac{(\log_2 n - 1)(\log_2 n - 2)}{2}$\\
                $\to n\log^2 n - 2n\log n - n\frac{(\log_2^2 n - 3\log_2 n + 2)}{2}$\\
                $\to n\log^2 n - 2n\log n - \frac{n\log_2^2 n}{2} + \frac{3\log_2 n}{2} - n$\\
                $\to \frac{n\log_2^2 n}{2} - 2n\log n + \frac{3\log_2 n}{2} - n$\\
                $\therefore$, $T_4(n) = \Theta(n\log n\log n)$\\
                \item $T_5(n) = \sqrt{n}T_5(\sqrt{n}) + n$, $n \geq 2$\\
                % We cannot use Master's Theorem here, as the form of the recurrence is not in the
                % form of $T(n) = aT(\frac{n}{b}) + f(n)$.\\
                % \begin{tabular}{c|c|c}
                %     Level  & Problem size    & Total time           \\
                %     \hline
                %     0      & n               & n                    \\
                %     1      & $n^{1/2}$       & $n^{1/2}n^{1/2}$     \\
                %     2      & $n^{1/2^2}$     & $n^{1/2^2}n^{1/2^2}$ \\
                %     \vdots & \vdots          & \vdots               \\
                %     k      & $n^{1/2^k} = 2$ & $n^{1/2^k}n^{1/2^k}$ \\
                % \end{tabular}\\
                % First, I'll solve for k\\
                % $n^{1/2^k} = 2$\\
                % $\to \log_2 n^{1/2^k} = \log_2 2 = 1$\\
                % $\to \frac{1}{2^k}\log_2 n = 1$\\
                % $\to \log_2 n = 2^k$\\
                % $\to \log_2 \log_2 n = k$\\
                I'll convert this recurrence into a different form to allow for using Master's
                Theorem\\
                Let $m = \log_2 n$; $n = 2^m$\\
                Then, $T_5(n) = \sqrt{n}T_5(\sqrt{n}) + n$ becomes $T_5(2^m) = 2^{m/2}T_5(2^{m/2}) + 2^m$\\
                Divide both sides by $2^m$\\
                This yields $\frac{T_5(2^m)}{2^m} = \frac{T_5(2^{m/2})}{2^{m/2}} + 1$\\
                We define $S(m) = \frac{T_5(2^m)}{2^m}$\\
                Now, the equation is equivalent to $S(m) = S(m/2) + 1$\\
                Using Master's Theorem, a = 1, b = 2, f(m) = 1\\
                $f(m) = \Theta(1)$\\
                $m^{\log_b a} = m^{\log_2 1} = m^0 = 1$\\
                $f(m) = \Theta(m^{\log_b a})$, so the recurrence is $\Theta(\log m)$\\
                Then, $S(m) = \Theta(\log m)$\\
                Since $S(m) = \frac{T_5(2^m)}{2^m}$, then $S(m)2^m = T_5(2^m)$\\
                Then, $T_5(2^m) = \Theta(2^m\log_2 m)$\\
                Then, $T_5(n) = \Theta(n \log n \log n)$\\
            \end{itemize}
            \item \maxpt{2}\\
            Both $T_4$ and $T_5$ are $\Theta(n \log n \log n)$\\
            They grow asymptotically slower than $T_1$, which is $\Theta(n^2)$\\
        \end{enumerate}
    \end{homeworkProblem}

    \newpage

    \begin{homeworkProblem}
        \textbf{(10 points)}
        \begin{enumerate}
            \item (4 points)\\
            Divide: Split array into 5 equal parts of size n/5.\\
            Combine: Merge the size n/5 sorted arrays back together into a size n sorted array.\\
            Split the array recursively into 5 equal parts, until the size of the array is 1.
            Then, merge the arrays back together.\\
            To merge the arrays, we will use 5 pointers to each of the 5 arrays, let them be
            size n/5.
            We then compare each of them to find the smallest element.
            Then, we will add the smallest element to the new array of size n, repeat until all
            pointers are at the end of their respective array.\\
            This yields a new sorted array of size n.\\
            We combine the sorted arrays to get the final sorted array.\\
            \item (2 points)\\
            Let each array be size n/5.
            We assume each of the arrays are sorted.\\
            Each element has at most 4 comparisons until some subarrays have no elements, so the
            total number of comparisons is at most 4n - $\sum_{i=1}^{5} i$ = 4n - 10.\\
            Example: [1, 6], [2, 7], [3, 8], [4, 9], [5, 10] | n = 10\\
            We compare 1, 2, 3, 4, 5 $\to$ 4 comparisons\\
            We compare 6, 2, 3, 4, 5 $\to$ 4 comparisons\\
            We compare 6, 7, 3, 4, 5 $\to$ 4 comparisons\\
            We compare 6, 7, 8, 4, 5 $\to$ 4 comparisons\\
            We compare 6, 7, 8, 9, 5 $\to$ 4 comparisons\\
            We compare 6, 7, 8, 9, 10 $\to$ 4 comparisons\\
            We compare 7, 8, 9, 10 $\to$ 3 comparisons\\
            We compare 8, 9, 10 $\to$ 2 comparisons\\
            We compare 9, 10 $\to$ 1 comparison\\
            We compare 10 $\to$ 0 comparisons\\
            Total comparisons = 30 = 4(10) - 10.\\
            \item (2 points)\\
            We split the array into 5 equal subarrays recursively so $T(n)$ becomes $5T(n/5)$.\\
            The combining step takes $\Theta(n)$ time.\\
            T(n) = 5T(n/5) + 4n - 10 with T(1) = 1\\
            \item (2 points)\\
            Using Master's theorem, a = 5, b = 5, f(n) = 4n - 10\\
            $f(n) = \Theta(n)$\\
            $n^{log_5 5} = n$.\\
            $f(n) = \Theta(n^{log_b a})$, so the recurrence is $\Theta(n\log n)$.\\
        \end{enumerate}
    \end{homeworkProblem}

    \newpage

    \begin{homeworkProblem}
        \maxpt{10}

        \begin{enumerate}
            \item Recursive Solution with Memoization
            \begin{lstlisting}
memo = [-1] * n
MR(i):
    if i > n:
        return 0
    if memo[i] != -1:
        return memo[i]
    MAX = max($p_i$ + MR(i + 1 + $c_i$), MR(i + 1))
    memo[i] = MAX
    return MAX
            \end{lstlisting}
            \item Iterative Solution
            \begin{lstlisting}
MT(low):
    memo[n] = 0
    for i in range(n, low - 1):
        memo[i] = max($p_i$ + memo[i + 1 + $c_i$], memo[i + 1])
    return memo[low]
            \end{lstlisting}
            \item Proof of Correctness\\
            Claim: MT(low) = OPT(low), in other words, MT returns the maximum points possible
            with tasks starting at time low\\
            Base Case: low = n\\
            MT(n) = $p_n$ if it's greater than 0, which is the maximum points possible.\\
            Inductive Hypothesis: Assume MT(i) = OPT(i) for all i $\geq$ k.\\
            Inductive Step: Prove MT(k - 1) = OPT(k - 1)\\
            By the inductive hypothesis, we know that MT(k) = OPT(k) and MT(i + $c_i$) = OPT(i + $
            c_i$)\\
            Case 1: It's better to not do the task k - 1\\
            In other words, OPT(k - 1) = OPT(k)\\
            If that's the case, our algorithm picks the maximum of the two options
            $p_{k-1}$ + MT(k + $c_{k-1}$) and MT(k), which would yield MT(k)\\
            Then, MT(k - 1) = MT(k)\\
            This yields MT(k - 1) = MT(k)\\
            By the inductive hypothesis, MT(k) = OPT(k), which is equivalent to OPT(k - 1) since
            the task is not worth taking\\
            Then, MT(k - 1) = OPT(k - 1)\\
            Then, case 1 of MT is optimal\\
            Case 2: It's better to do the task k - 1\\
            In other words, OPT(k - 1) = $p_{k-1}$ + OPT(k + $c_{k-1}$)\\
            If that's the case, our algorithm picks the maximum of the two options
            $p_{k-1}$ + MT(k + $c_{k-1}$) and MT(k), which would yield $p_{k-1}$ + MT(k + $c_{k-1}$)\\
            Then, MT(k - 1) = $p_{k-1}$ + MT(k + $c_{k - 1}$)\\
            By the inductive hypothesis, MT(k + $c_{k-1}$) = OPT(k + $c_{k-1}$)\\
            Add $p_{k-1}$ to both sides, we get $p_{k-1}$ + MT(k + $c_{k-1}$) = $p_{k-1}$ + OPT(k
            + $c_{k-1}$) = OPT(k - 1)\\
            Then, MT(k - 1) = OPT(k - 1)\\
            Then, case 2 of MT is optimal\\
            Therefore, MT(k - 1) = OPT(k - 1)\\
            Therefore, MT(low) = OPT(low)\\
            Therefore, the iterative solution is correct and optimal.\\
            Since the iterative solution iterates through the array once, with a constant time
            per iteration, it's $O(n)$.\\
        \end{enumerate}
    \end{homeworkProblem}
\end{document}
%%% Local Variables:
%%% mode: latex
%%% TeX-master: t
%%% End:
