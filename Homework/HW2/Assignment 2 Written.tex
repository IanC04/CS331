\documentclass[12pt]{article}
\usepackage{amsmath,amsfonts,amsthm,amssymb}
\usepackage{setspace}
\usepackage{fancyhdr}
\usepackage{lastpage}
\usepackage{verbatim}
\usepackage{extramarks}
\usepackage[ruled,vlined]{algorithm2e}
\usepackage{chngpage}
\usepackage{soul,color}
\usepackage{graphicx,float,wrapfig}
\usepackage{ listings}
\lstset{
    basicstyle=\ttfamily,
    mathescape
}
\usepackage{enumitem}
\newcommand{\Class}{ \normalsize CS 331: Algorithms and Complexity (Spring 2024)\\
\small    Unique Number: 52765, 52770
}


%\newcommand{\ClassInstructor}{Fares}
\def\changemargin#1#2{\list{}{\rightmargin#2\leftmargin#1}\item[]}
\let\endchangemargin=\endlist

% Homework Specific Information. Change it to your own

\newcommand{\Title}{Assignment 2}
\newcommand{\DueDate}{Thursday, 1 Febrauary, by 11.59pm}

\newcommand{\StudentName}{}
\newcommand{\StudentClass}{}
\newcommand{\StudentNumber}{}

% In case you need to adjust margins:
\topmargin=-0.45in      %
\evensidemargin=0in     %
\oddsidemargin=0in      %
\textwidth=6.5in        %
\textheight=9.0in       %
\headsep=0.25in         %

% Setup the header and footer
\pagestyle{fancy}                                                       %
\lhead{\StudentName}                                                 %
\chead{\Title}  %
\rhead{\firstxmark}                                                     %
\lfoot{\lastxmark}                                                      %
\cfoot{}                                                                %
\rfoot{Page\ \thepage\ of\ \protect\pageref{LastPage}}                          %
\renewcommand\headrulewidth{0.4pt}                                      %
\renewcommand\footrulewidth{0.4pt}                                      %

%%%%%%%%%%%%%%%%%%%%%%%%%%%%%%%%%%%%%%%%%%%%%%%%%%%%%%%%%%%%%
% Some tools
\newcommand{\enterProblemHeader}[1]{\nobreak\extramarks{#1}{#1 continued on next page\ldots}\nobreak%
\nobreak\extramarks{#1 (continued)}{#1 continued on next page\ldots}\nobreak}%
\newcommand{\exitProblemHeader}[1]{\nobreak\extramarks{#1 (continued)}{#1 continued on next page
\ldots}\nobreak%
\nobreak\extramarks{#1}{}\nobreak}%

\newcommand{\homeworkProblemName}{}%
\newcounter{homeworkProblemCounter}%
\newenvironment{homeworkProblem}[1][Problem \arabic{homeworkProblemCounter}]%
{\stepcounter{homeworkProblemCounter}%
\renewcommand{\homeworkProblemName}{#1}%
\section*{\homeworkProblemName}%
\enterProblemHeader{\homeworkProblemName}}%
{\exitProblemHeader{\homeworkProblemName}}%

\newcommand{\homeworkSectionName}{}%
\newlength{\homeworkSectionLabelLength}{}%
\newenvironment{homeworkSection}[1]%
{% We put this space here to make sure we're not connected to the above.

    \renewcommand{\homeworkSectionName}{#1}%
    \settowidth{\homeworkSectionLabelLength}{\homeworkSectionName}%
    \addtolength{\homeworkSectionLabelLength}{0.25in}%
    \changetext{}{-\homeworkSectionLabelLength}{}{}{}%
    \subsection*{\homeworkSectionName}%
    \enterProblemHeader{\homeworkProblemName\ [\homeworkSectionName]}}%
    {\enterProblemHeader{\homeworkProblemName}%

    % We put the blank space above in order to make sure this margin
    % change doesn't happen too soon.
    \changetext{}{+\homeworkSectionLabelLength}{}{}{}}%

\newcommand{\Answer}{\ \\\textbf{Answer:} }
\newcommand{\Acknowledgement}[1]{\ \\{\bf Acknowledgement:} #1}

%%%%%%%%%%%%%%%%%%%%%%%%%%%%%%%%%%%%%%%%%%%%%%%%%%%%%%%%%%%%%


%%%%%%%%%%%%%%%%%%%%%%%%%%%%%%%%%%%%%%%%%%%%%%%%%%%%%%%%%%%%%
% Make title
\title{\textmd{\bf \Class\\ \Title}\\\vspace{0.1in}\small{Due\ on\ \DueDate}}
\date{}
\author{\textbf{\StudentName}\ \ \StudentClass\ \ \StudentNumber}
%%%%%%%%%%%%%%%%%%%%%%%%%%%%%%%%%%%%%%%%%%%%%%%%%%%%%%%%%%%%%

\setlist[enumerate,1]{label=\bfseries(\alph*)}
\newcommand{\maxpt}[1]{\ifthenelse{\equal{#1}{1}}{\textbf{(#1 pt)}}{\textbf{(#1 pts)}}}

\begin{document}
    \maketitle \thispagestyle{empty}



    \begin{homeworkProblem}[Problem 1: Short Answers Section]
        For this section, restrict answers to no more than a few sentences.
        Most answers can be expressed in a single sentence. Unless otherwise
        stated, briefly justify. No proofs are necessary for this section.
        \begin{enumerate}
            \item
            True, since at first there is a root node with no edges. Every subsequent node adds one
            edge.
            \item
            False, there may be cycles.
            However, DAG $\rightleftarrows$ Topological Ordering.
            \item
            True, BFS would iterate every edge set of every vertex, so $\lvert V\rvert\cdot\lvert V\rvert$ edges.
            \item
            True, iterate through the node's row of the matrix, which has $\lvert V\rvert$ elements.
            \item
            False, DFS will in general output deeper trees, but not always.
            Counterexample: a tree with only a root node.
            \item
            True, if there were multiple paths, then that means there's a cycle.
            Trees have no cycles.
        \end{enumerate}
    \end{homeworkProblem}

    \begin{homeworkProblem}
        \begin{enumerate}
            [=(\alph*)]
            \item
            True, we can alternate the colors in each layer of the tree. Then, every edge in the
            tree will be touching a node on layer$_n$ and one on layer$_{n+1}$, which are
            different colors. An algorithm is to run BFS starting from the tree's root.
            \item
            Nodes can't connect to other nodes on the same layer, as that would create an odd
            length cycle. They also can't connect to nodes an even number of layers away, as that
            would connect two nodes of the same color. So, they can only connect to nodes an odd
            number of layers away. We know that each node of a color can connect to every node of
            the other color and that a tree has n - 1 edges.
            \begin{verbatim}
def add_edges(graph):
    black = 0
    white = 0
    Initialize queue as storing nodes with their associated depth
    Add root node to queue with depth 0
    while queue is not empty:
        node, depth = queue.poll()
        if depth % 2 == 0:
            black += 1
        else:
            white += 1
        for neighbor in node.neighbors:
                queue.add(neighbor, depth + 1)
    return black * white - (n - 1)
            \end{verbatim}
        \end{enumerate}
    \end{homeworkProblem}

    \begin{homeworkProblem}
        I would run a breadth-first search, which has O($\mid V\mid +\mid E\mid$).\newline
        \begin{proof}
            \textbf{Inductive hypothesis:} At step k, all possible routes to nodes of distance k are
            stored in the path\_count variable of each node and the queue stores all nodes k
            distance away from the s.\newline
            \textbf{Initialization:} At step 0, the path\_count of the start node is 1, and all
            other nodes are unvisited. The queue stores only s.\newline
            \textbf{Maintenance:} At step k, we have a list of nodes of distance k. For each
            node, it stores the distance from s and the number of paths to it. For each node, we
            iterate through its edges. If the edge's node has not been visited, we set its
            distance to k+1, and by BFS, it's the shortest distance. We also set its
            path\_count to the current node's + the node's path\_count. If the edge's node has
            been visited, we don't need to add it again to the queue, but we do need to
            increment its path\_count by the current node's path\_count. We do this for all nodes
            of distance k, and then we increment k.\newline
            Now the queue contains all nodes of distance k+1 and the number of unique paths to it
            .\newline
            \textbf{Termination:} The algorithm terminates when the target node's distance is
            greater than the current node's distance. At this point, we return the target node's
            path\_count.
        \end{proof}
        \begin{verbatim}
def bfs(graph, start):
    Initialize all nodes to path_count = 0 and distance = infinity
    Initialize s to path_count = 1 and distance = 0
    Initialize queue to list(s)
    while queue is not empty:
        node = queue.poll()
        if node.distance > t.distance:
            break
        for neighbor in node.neighbors:
            if neighbor.distance == infinity:
                neighbor.distance = node.distance + 1
                neighbor.path_count = node.path_count
                queue.add(neighbor)
            else if neighbor.distance == node.distance + 1:
                neighbor.path_count += node.path_count
    return t.path_count
        \end{verbatim}
    \end{homeworkProblem}

\end{document}
