\documentclass[12pt]{article}
\setlength\parindent{0pt}
\usepackage{amsmath,amsfonts,amsthm,amssymb}
\usepackage{setspace}
\usepackage{fancyhdr}
\usepackage{lastpage}
\usepackage{extramarks}
\usepackage[ruled,vlined]{algorithm2e}
\usepackage{chngpage}
\usepackage{soul,color}
\usepackage{graphicx,float,wrapfig}
\usepackage{ listings}
\newcommand{\Class}{ \normalsize CS 331: Algorithms and Complexity (Spring  2024)\\
	\small	Unique numbers:  50930, 50935 50940, 50945
}
%\newcommand{\ClassInstructor}{Fares}
% Homework Specific Information. Change it to your own
\newcommand{\Title}{Classwork 1}
\newcommand{\DueDate}{}
\newcommand{\StudentName}{}
\newcommand{\StudentClass}{}
\newcommand{\StudentNumber}{}

% In case you need to adjust margins:
\topmargin=-0.45in      %
\evensidemargin=0in     %
\oddsidemargin=0in      %
\textwidth=6.5in        %
\textheight=9.0in       %
\headsep=0.25in         %

% Setup the header and footer
\pagestyle{fancy}                                                       %
\lhead{\StudentName}                                                 %
\chead{\Title}  %
\rhead{\firstxmark}                                                     %
\lfoot{\lastxmark}                                                      %
\cfoot{}                                                                %
\rfoot{Page\ \thepage\ of\ \protect\pageref{LastPage}}                          %
\renewcommand\headrulewidth{0.4pt}                                      %
\renewcommand\footrulewidth{0.4pt}                                      %

%%%%%%%%%%%%%%%%%%%%%%%%%%%%%%%%%%%%%%%%%%%%%%%%%%%%%%%%%%%%%
% Some tools
\newcommand{\enterProblemHeader}[1]{\nobreak\extramarks{#1}{#1 continued on next page\ldots}\nobreak%
	\nobreak\extramarks{#1 (continued)}{#1 continued on next page\ldots}\nobreak}%
\newcommand{\exitProblemHeader}[1]{\nobreak\extramarks{#1 (continued)}{#1 continued on next page\ldots}\nobreak%
	\nobreak\extramarks{#1}{}\nobreak}%

\newcommand{\homeworkProblemName}{}%
\newcounter{homeworkProblemCounter}%
\newenvironment{homeworkProblem}[1][Problem \arabic{homeworkProblemCounter}]%
{\stepcounter{homeworkProblemCounter}%
	\renewcommand{\homeworkProblemName}{#1}%
	\section*{\homeworkProblemName}%
	\enterProblemHeader{\homeworkProblemName}}%
{\exitProblemHeader{\homeworkProblemName}}%

\newcommand{\homeworkSectionName}{}%
\newlength{\homeworkSectionLabelLength}{}%
\newenvironment{homeworkSection}[1]%
{% We put this space here to make sure we're not connected to the above.
	
	\renewcommand{\homeworkSectionName}{#1}%
	\settowidth{\homeworkSectionLabelLength}{\homeworkSectionName}%
	\addtolength{\homeworkSectionLabelLength}{0.25in}%
	\changetext{}{-\homeworkSectionLabelLength}{}{}{}%
	\subsection*{\homeworkSectionName}%
	\enterProblemHeader{\homeworkProblemName\ [\homeworkSectionName]}}%
{\enterProblemHeader{\homeworkProblemName}%
	
	% We put the blank space above in order to make sure this margin
	% change doesn't happen too soon.
	\changetext{}{+\homeworkSectionLabelLength}{}{}{}}%

\newcommand{\Answer}{\ \\\textbf{Answer:} }
\newcommand{\Acknowledgement}[1]{\ \\{\bf Acknowledgement:} #1}

%%%%%%%%%%%%%%%%%%%%%%%%%%%%%%%%%%%%%%%%%%%%%%%%%%%%%%%%%%%%%


%%%%%%%%%%%%%%%%%%%%%%%%%%%%%%%%%%%%%%%%%%%%%%%%%%%%%%%%%%%%%
% Make title
\title{\textmd{\bf \Class\\ \Title}\\\vspace{0.1in}\small{}}
\date{}
\author{\textbf{\StudentName}\ \ \StudentClass\ \ \StudentNumber}
%%%%%%%%%%%%%%%%%%%%%%%%%%%%%%%%%%%%%%%%%%%%%%%%%%%%%%%%%%%%%

\begin{document}
		\maketitle \thispagestyle{empty}
		
		%%%%%%%%%%%%%%%%%%%%%%%%%%%%%%%%%%%%%%%%%%%%%%%%%%%%%%%%%%%%%
		% Begin edit from here

	
\begin{homeworkProblem}[]


\textbf{(a)} Prove that $n^3-n$ is divisible by 3 for all positive integers.\\
\textbf{Sol.}
\\
\textbf{Base case:} $n=1$,
$1^3-1$ is divisible by 3, 0 divisible by 3. \\
\\
\textbf{Induction hypothesis:} Assume that the formula holds for $n=k$, i.e.,
$k^3-k$ is divisible by 3 \\
\\
\textbf{Induction Step:} show that the formula holds for $n=k+1$, i.e.,
$(k+1)^3-(k+1)$ is divisible by 3

\begin{align*}
(k+1)^3 - (k+1) &  = (k^3+3k^2+3k+1) - (k+1) &\\
&= (k^3-k) + 3k^2 + 3k &\\
&= (k^3-k) + 3(k^2+k) &\\
&\text{But we know that} (k^3-k) \text{ is divisible by 3 by induction hypothesis, and} &\\
&3(k^2+k) \text{ is divisible by 3 } &\\
&\text{Conclusion:} (k+1)^3-(k+1) \text{ is divisible by 3} &\\
& \text{Since if $a$ is divisisble by $b$ and $c$ is divisible by $b$, then $a+c$ is divisible by $b$}
\end{align*}
\newpage
\textbf{(b)} The Fibonacci numbers $F_n \ge$0 are the numbers of a famous sequence defined by\\
$F_0$ = 0 \\
$F_1$ = 1 \\
\ldots \\
$F_n$ = $F_{n-1}$ + $F_{n-2}$.\\

\textbf{Prove} that for all $n \ge$ 0, $F_n < 2^n$.\\
\textbf{Sol.}
\\
Define $P(n)$ = $F_n < 2^n$.\\
\\
\textbf{Base case:} We need two base cases since the definition of $F_n$ requires two values.\\
\textbf{Base Case 1}: $n$ = 0. $F_0$ = 0, which is less than $2^0$ = 1. \\
\textbf{Base Case 2}: $n$ = 1. $F_1$ = 0, which is less than $2^1$ = 2.\\

\textbf{Induction hypothesis:} Assume that the formula holds for $n=k$, i.e.,
$P(0)$, $P(1)$, \ldots, $P(k)$ are all true. \\
\\
\textbf{Induction Step:} show that the formula holds for $n=k+1$, i.e.,
$P(k+1)$, or $F_{k+1} < 2^{k+1}$. \\
$F_{k+1}$ = $F_k$ + $F_{k-1} < 2^k + 2^{k-1} < 2^k + 2^k = 2(2^k) = 2^{k+1}$.\\\\
By strong induction, for all $n \ge 0$, $F_n < 2^n$. 


\textbf{(c)} You were asked to provide an algorithm that finds the maximum value in a given array, $A$, with $n$ elements. The algorithm is supposed to scan through each location in the array and report the maximum value seen so far. Algorithm \ref{alg:Q1b} is purported to solve this problem. Show that this is indeed the case by providing a mathematical proof of correctness for Algorithm \ref{alg:Q1b}. I.e., show that it finds the maximum value in $A$. Note: the array index starts from 0.

\begin{algorithm}[!h]\label{alg:Q1b}
	\caption{Find maximum in array}
	Initially $index = 0$ and $\max = A[index]$\;
	\While {\text{index} $< n$}
	{  \If {$A[\text{index}] > \max$}
		{$max = A[index]$\;
		}
	Increase $index$ by $1$\;
	}
	report $max$\;
\end{algorithm}

\textbf{Sol:}
\textbf{Base case:} If $n=1$, then the body of the while loop doesn't get executed and we report the maximum value as $A[0]$.\\

\textbf{Induction hypothesis:} Assume that max stores the maximum value of $A[0..n-2]$ (which covers $n$-1 elements of the array $A$). \\

\textbf{Induction step:} Show that $\max$ stores the maximum value of $A[0..n-1]$ (which covers $n$ elements of the array $A$). \\

We know from the induction hypothesis that at the $(n-1)$-th iteration, max stores the maximum value of $A[0..n-2]$. In the $n$-th and final iteration, if $A[n-1] > \max$, then $\max$ will store $A[n-1]$. Otherwise, $\max$ will keep the maximum of $A[0..n-2]$ since $A[n-1]$ is not bigger. Therefore, upon termination of the final iteration, $\max$ will store the maximum of $A[0..n-1]$.



\end{homeworkProblem}


\end{document}